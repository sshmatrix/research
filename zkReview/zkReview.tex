\documentclass[a4paper,10pt]{article}

% ------------------------------------
% packages
\usepackage[a4paper,margin=1in,footnotesep=2.2\baselineskip]{geometry}
\usepackage{multicol}
\usepackage{xcolor}
\usepackage{framed}
\usepackage{emoji}
\usepackage[most]{tcolorbox}
\usepackage{fancyhdr}
\usepackage[tracking=true]{microtype}
\usepackage{ragged2e}
\usepackage{listings}
\usepackage{color}
\usepackage{pagecolor}
\usepackage{pdftexcmds}

% ------------------------------------
% checksum = SHA-1
\makeatletter
\ifx\pdf@filemdfivesum\undefined\def\pdf@filemdfivesum#{\mdfivesum file}\fi
\let\filesum\pdf@filemdfivesum
\makeatother

% ------------------------------------
% color definitions
\definecolor{armygreen}{rgb}{0.14, 0.71, 0.15}
\definecolor{darkgreen}{rgb}{0.08, 0.48, 0.18}
\definecolor{darkred}{rgb}{0.86, 0.153, 0.153}
\definecolor{azure}{rgb}{0.0, 0.5, 1.0}
\definecolor{bole}{rgb}{0.82, 0.57, 0.22}
\definecolor{dkgreen}{rgb}{0,0.6,0}
\definecolor{gray}{rgb}{0.5,0.5,0.5}
\definecolor{mauve}{rgb}{0.58,0,0.82}
\definecolor{light-gray}{gray}{0.95}
\definecolor{bg}{HTML}{fcfcfa}
\definecolor{bt}{HTML}{ffebe6}

% ------------------------------------
% code styling
\definecolor{shadecolor}{RGB}{180,180,180}
\newcommand{\code}[1]{\colorbox{shadecolor!30}{\mono{#1}}}
\colorlet{Gray}{gray!20!}
\tcbset{on line, arc=1pt, leftrule=0.25pt,rightrule=0.25pt,toprule=0.25pt,bottomrule=0.25pt,
	boxsep=3pt, left=0pt,right=0pt,top=0pt,bottom=0pt,
	colframe=white,colback=Gray,  
	highlight math style={enhanced}
}
\lstdefinelanguage{Solidity}{
  keywords={bool, true, false, return, address, bytes32, bytes4, bytes1, bytes, uint256, uint8, uint, string, if, while, else, case, break},
  keywordstyle=\color{blue}\bfseries,
  ndkeywords={function,struct, mapping, export, throw, implements, import, this},
  ndkeywordstyle=\color{darkgreen}\bfseries,
  identifierstyle=\color{black},
  sensitive=false,
  comment=[l]{//},
  morecomment=[s]{/*}{*/},
  commentstyle=\color{green}\mono,
  stringstyle=\color{orange}\mono,
  morestring=[b]',
  morestring=[b]",
  mathescape=true,
  literate={=>}{$\rightarrow{}$}{1}
}
\lstset{
  backgroundcolor=\color{light-gray},
  language=Solidity,
  aboveskip=3mm,
  belowskip=3mm,
  showstringspaces=false,
  columns=flexible,
  basicstyle={\small\mono},
  numbers=none,
  numberstyle=\tiny\color{gray},
  keywordstyle=\color{blue},
  commentstyle=\color{dkgreen},
  stringstyle=\color{mauve},
  breaklines=true,
  breakatwhitespace=true,
  tabsize=3
}

% ------------------------------------
% fonts
\newfontfamily\pro[Path=./]{SFMono.ttf}
\newfontfamily\pbold[Path=./]{SFMonoBold.ttf}
\newfontfamily\mono[Path=./]{SFMono.ttf}
\newfontfamily\mbold[Path=./]{SFMonoBold.ttf}

% ------------------------------------
% heading font-size
\usepackage{sectsty}
\usepackage{fontspec}
\sectionfont{\fontsize{12}{15}\selectfont}
\usepackage[utf8]{inputenc}

% ------------------------------------
% footnote positioning
\usepackage[hang,flushmargin,bottom]{footmisc} 

% ------------------------------------
% bibliography
\usepackage[colorlinks=true,
            linkcolor=blue,
            urlcolor=blue,
            citecolor=blue,
            pdfauthor={sshmatrix},
            pdftitle={Helix2 Protocol},
            pdfsubject={Link Service and Protocol},
            pdfkeywords={ethereum, account, abstraction, link, name, decentralised, distributed},
            pdfproducer={sshmatrix},
            pdfcreator={sshmatrix}]{hyperref}
            
% ------------------------------------
% blank footnote
\newcommand\blfootnote[1]{%
	\begingroup
	\renewcommand\thefootnote{}\footnote{#1}%
	\addtocounter{footnote}{-1}%
	\endgroup
}

% ------------------------------------
% header
\pagestyle{fancy}
\fancyhf{}
\lfoot{\footnotesize \mono{\#}\tcbox{\mono{\filesum{\jobname}}}}
\begin{document}
\setcounter{footnote}{0}
\newpage
\topskip15pt

\fancyhead[L]{\footnotesize \mono{author}:\tcbox{\mono{sshmatrix}}}
\fancyhead[R]{{\footnotesize \mono{\href{https://github.com/}{github.com/}}}}
\fancyhead[C]{{\footnotesize \emoji{dna}}}
\fancyfoot[C]{{\footnotesize \mono{\thepage/--}}}
\fancyfoot[R]{{\footnotesize \mono{\today} \emoji{dna}}}

\begin{center}
	\textbf{\Large\pbold{\textls[-40]{Intuitive Interpretation of Non-Interactive Zero-Knowledge Cryptography}}}\\
	\vspace{0.075in}
	\textls[-50]{\mono{A jargon-free approach to understanding zk-SNARKs and zk-STARKs}}\linebreak\linebreak
	\vspace{-0.175in}
	\textls[0]{\mono{Avneet Singh}}\linebreak\linebreak
	\textls[0]{\mono{Interplanetary Company UG}}\linebreak
	\textls[0]{\mono{\href{mailto:sshmatrix@proton.me}{sshmatrix@proton.me}}}\linebreak
\end{center}
\begin{center}
	\textbf{\Large\pbold{ABSTRACT}}\linebreak\linebreak
	\textls[-50]{\mono{zk-SNARKs and zk-STARKs are relatively new concepts in cryptography, yet they are being touted as the next forefront in modern and future crypto tech. In blockchain space specifically, there is great interest in these subfields in the context of zk-Rollups to Layer 1 blockchains such as Ethereum, or as standalone decentralised ledgers with high rates of transactions per second (TPS), e.g. Aztec Network (zk-STARK), zkSync, Loopring, ZCash (zk-SNARKs) etc. Despite their great importance in cryptography, it is unfortunately difficult to understand zk-SNARKs and zk-STARKs due to limited literature and conceivably difficult mathematics conveyed through intensive jargon. This paper is an attempt to introduce zero-knowledge (zk) cryptography in an intuitive manner to garden-variety mathematicians, physicists, curious blockchain developers and perhaps even cryptographers.}}
\end{center}
\vspace{0.2in}
\begin{flushleft}
	\textbf{\Large\pbold{INTRODUCTION}}\linebreak\linebreak
	\textls[-50]{\mono{}}
\end{flushleft}

\begin{flushleft}
	\textbf{\Large\pbold{REFERENCES}}\linebreak\linebreak
		\textls[-50]{\mono{
			
		}}
\end{flushleft}
\begin{flushright}
	\textbf{\large\pbold{METADATA}}\linebreak\linebreak
	\textls[-50]{\mono{
		\mono{Github: }\tcbox{}\linebreak
		\mono{Contracts: }\tcbox{}\linebreak
		\mono{Source: }\tcbox{}\linebreak
		\mono{SHA-1 Checksum: }\tcbox{}\linebreak
		\mono{Date: }\tcbox{\mono{\today}}\linebreak
	}}
\end{flushright}
\end{document}
